% Compiled and edited by Brian Bi
\documentclass[10pt]{extarticle}
\setlength{\parindent}{0.0in}
\usepackage{amsmath}
\usepackage{multicol}
\usepackage[landscape,letterpaper,twoside=false,top=15mm,bottom=15mm,left=10mm,right=10mm]{geometry}
\pagestyle{myheadings}
\markright{}
\usepackage{listings}
\usepackage{color}
\lstset{
	tabsize=4,
    basicstyle=\ttfamily\scriptsize,
    %upquote=true,
    aboveskip={1.5\baselineskip},
    columns=fixed,
    showstringspaces=false,
    extendedchars=true,
    breaklines=true,
    prebreak = \raisebox{0ex}[0ex][0ex]{\ensuremath{\hookleftarrow}},
    frame=single,
    rulecolor=\color[rgb]{0.75,0.75,0.75},
    showtabs=false,
    showspaces=false,
    showstringspaces=false,
    keywordstyle=\color[rgb]{0,0,1},
    commentstyle=\color[rgb]{0.133,0.545,0.133},
    stringstyle=\color[rgb]{0.627,0.126,0.941},
}

\begin{document}
% Disable balancing of columns on last page, which is ugly
\begin{multicols*}{2}
% Want compact table of contents, but normal spacing between paragraphs later on
\setlength{\parskip}{0.0in}
\tableofcontents
\setlength{\parskip}{0.1in}
% New codebook
\section{Flow and matching}

\subsection{Max flow (Dini\'c)} % Stanford
\lstinputlisting[language=c++]{dinic.cpp}

\subsection{Min-cost max-flow (successive shortest paths)}
% Frank Chu and Igor Naverniouk (modified by Brian Bi)
\lstinputlisting[language=c++]{mcmf4.cpp}

\subsection{Max bipartite matching} % Stanford
\lstinputlisting[language=c++]{bipartite.cpp}




\section{Geometry}

\subsection{Miscellaneous Geometry} % Stanford
\lstinputlisting[language=c++]{geom-2d.cpp}


\subsection{Convex hull} % Brian Bi
\lstinputlisting[language=c++]{monotone.cpp}

\subsection{Minimum Enclosing Disk (Welzl's Algorithm)} 
\lstinputlisting[language=c++]{welzl.cpp}

\subsection{Pick's Theorem (Text)} % Wesley May
For a polygon with all vertices on lattice points, $A = i + b/2 - 1$, where $A$
is the area, $i$ is the number of lattice points strictly within the polygon,
and $b$ is the number of lattice points on the boundary of the polygon. (Note,
there is no generalization to higher dimensions)

\section{Math Algorithms}

\subsection{Sieve of Eratosthenes} % Jimmy Mårdell (Yarin)
\lstinputlisting[language=c++]{yarin.cpp}

\subsection{Modular arithmetic and linear Diophantine solver} % Stanford
\lstinputlisting[language=c++]{modular.cpp}

\subsection{Gaussian elimination} % Stanford
\lstinputlisting[language=c++]{gaussian.cpp}


\subsection{Solving linear systems (Text)} % Brian Bi
To solve a general system of linear equations, put it into matrix form and
compute the reduced row echelon form. For example,
\begin{align*}2x + y &= 5 \\ 3x + 2y &= 6\end{align*}
corresponds to the matrix
\[ \left[ \begin{array}{cc|c} 2 & 1 & 5 \\ 3 & 2 & 6 \end{array} \right] \]
with RREF
\[ \left[ \begin{array}{cc|c} 1 & 0 & 4 \\ 0 & 1 & -3 \end{array} \right] \]
After row reduction, if any row has a 1 in the rightmost column and 0
everywhere else, then the system is inconsistent and has no solution.
Otherwise, to find a solution, set the variable corresponding to the leftmost 1
in each column equal to the corresponding value in the rightmost column, and
set all other variables to 0. Ignore rows consisting entirely of 0. The
solution is unique iff the rank of the matrix equals the number of variables.

\subsection{Fast Fourier transform (FFT)} % Stanford
\lstinputlisting[language=c++]{fft.cpp}



\subsection{Fast factorization (Pollard rho) and primality testing
(Rabin--Miller)} % Qiyu Zhu
\lstinputlisting[language=c++]{pollard-rho.cpp}

\subsection{Euler's Totient} % Andre Hahn Pereira
\lstinputlisting[language=c++]{totient.cpp}

\section{Graphs}

\subsection{Strongly connected components} % Stanford (modified by Brian Bi)
\lstinputlisting[language=c++]{scc.cpp}

\subsection{Bridges} % Andre Hahn Pereira
\lstinputlisting[language=c++]{bridges.cpp}

\subsection{Bellman Ford}
\lstinputlisting[language=c++]{BellmanFord.cpp}

\subsection{Dijkstra}
\lstinputlisting[language=c++]{dijkstra.cpp}

\subsection{Floyd Warshall}
\lstinputlisting[language=c++]{FloydWarshall.cpp}

\subsection{Kruskal using MST}
\lstinputlisting[language=c++]{KruskalMST.cpp}

\subsection{Centroid Decompostion}
\lstinputlisting[language=c++]{centroid_decomposition.cpp}

\subsection{Lowest Common Ancestor} 
\lstinputlisting[language=c++]{LCA.cpp}


\section{Data Structures}

\subsection{Suffix arrays} % Stanford
\lstinputlisting[language=c++]{suffix-array.cpp}

\subsection{Palindromic Tree} % Stanford
\lstinputlisting[language=c++]{palindromictree.cpp}

\subsection{Binary Indexed Tree (BIT) / Fenwick Tree} % Brian Bi
\lstinputlisting[language=c++]{BIT.cpp}

\subsection{Binary Indexed Tree (2D BIT)} % Brian Bi
\lstinputlisting[language=c++]{2dBIT.cpp}

\subsection{Segment Tree with Lazy Propagation} % Brian Bi
\lstinputlisting[language=c++]{segtree.cpp}

\subsection{Sparse Table RMQ} % Brian Bi
\lstinputlisting[language=c++]{sparsetablermq.cpp}

\subsection{Trie} % Brian Bi
\lstinputlisting[language=c++]{Trie.cpp}

\section{Miscellaneous}

\subsection{Knuth--Morris--Pratt (KMP)} % Stanford
\lstinputlisting[language=c++]{KMP.cpp}

\subsection{2-SAT} % Brian Bi
\lstinputlisting[language=c++]{2sat.cpp}

\subsection{LCS} % Brian Bi
\lstinputlisting[language=c++]{LCS.cpp}

\subsection{LIS} % Brian Bi
\lstinputlisting[language=c++]{lis.cpp}


\subsection{Longest palindromic substring(Manacher)} % Brian Bi
\lstinputlisting[language=c++]{manacher.cpp}

\subsection{Z and Prefix Function} % Brian Bi
\lstinputlisting[language=c++]{Z&PF.cpp}

\subsection{Ternary Search} % Brian Bi
\lstinputlisting[language=c++]{TernarySearch.cpp}

\subsection{C ++ Template} % Brian Bi
\lstinputlisting[language=c++]{template.cpp}

\subsection{C ++ Build System} % Brian Bi
\lstinputlisting[language=c++]{C++build_system}

\subsection{Java BigInt} % Brian Bi
\lstinputlisting[language=Java]{BigIntJava.java}

\subsection{Java Template} % Brian Bi
\lstinputlisting[language=Java]{javatemplate.java}

\subsection{Python Fast I/O} % Brian Bi
\lstinputlisting[language=Python]{FastIOPython.py}


\end{multicols*}
\end{document}
